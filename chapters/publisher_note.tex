
% ----- 出版者的话 -----
\chapter*{出版者的话}
\addcontentsline{toc}{chapter}{出版者的话}
\markboth{出版者的话}{}
% \lipsum[7-9]| 出版者的话

文艺复兴以来,源远流长的科学精神和逐步形成的学术规范,使西方国家在自然科学的
各个领域取得了垄断性的优势;也正是这样的优势,使美国在信息技术发展的六十多年间名
家辈出、独领风骚。在商业化的进程中,美国的产业界与教育界越来越紧密地结合,计算机
学科中的许多泰山北斗同时身处科研和教学的最前线,由此而产生的经典科学著作,不仅擘
划了研究的范畴,还揭示了学术的源变,既遵循学术规范,又自有学者个性,其价值并不会
因年月的流逝而减退。

近年,在全球信息化大潮的推动下,我国的计算机产业发展迅猛,对专业人才的需求日
益迫切。这对计算机教育界和出版界都既是机遇,也是挑战;而专业教材的建设在教育战略
上显得举足轻重。在我国信息技术发展时间较短的现状下,美国等发达国家在其计算机科学
发展的几十年间积淀和发展的经典教材仍有许多值得借鉴之处。因此,引进一批国外优秀计
算机教材将对我国计算机教育事业的发展起到积极的推动作用,也是与世界接轨、建设真正
的世界一流大学的必由之路。

机械工业出版社华章公司较早意识到“出版要教育服务”。自1998年开始,我们
就将工作重点放在了遴选、移译国外优秀教材上。经过多年的不懈努力,我们与 Pearson,
McGraw-Hill, Elsevier,MIT,John Wiley \& Sons,Cengage 等世界著名出版公司建立了良
好的合作关系,从他们现有的数百种教材中甄选出 Andrew S. Tanenbaum,Bjarne Stroustrup,
Brain W. Kernighan,Dennis Ritchie,Jim Gray, Afred V. Aho, John E. Hopcroft, Jeffrey D.
Ullman,Abraham Silberschatz, William Stallings, Donald E. Knuth, John L. Hennessy, Larry L.
Peterson 等大师名家的一批经典作品,以“计算机科学丛书”总称出版,供读者学习、研
究及珍藏。大理石纹理的封面,也正体现了这套丛书的品位和格调。

“计算机科学丛书”的出版工作得到了国内外学者的鼎力相助,国内的专家不仅提供了
中肯的选题指导,还不辞劳苦地担任了翻译和审校的工作;而原书的作者也相当关注其作品
在中国的传播,有的还专门为其书的中译本作序。迄今,“计算机科学丛书”已经出版了近
两百个品种,这些书籍在读者中树立了良好的口碑,并被许多高校采用为正式教材和参考书
籍。其影印版“经典原版书库”作为妹妹篇也被越来越多实施双语教学的学校所采用。
权威的作者、经典的教材、一流的译者、严格的审校、精细的编辑,这些因素使我们
的图书有了质量的保证。随着计算机科学与技术专业学科建设的不断完善和教材改革的逐渐
深化,教育界对国外计算机教材的需求和应用都将步人一个新的阶段,我们的目标是尽善尽
美,而反馈的意见正是我们达到这一终极目标的重要帮助。华章公司欢迎老师和读者对我们
的工作提出建议或给予指正,我们的联系方法如下:

华章网站:www.hzbook.com

电子邮件:hzjsj@hzbook.com

联系电话:(010)88379604

联系地址:北京市西城区百万庄南街1号

邮政编码:100037

HZ BOOKS

华章教育

华章科技图书出版中心
