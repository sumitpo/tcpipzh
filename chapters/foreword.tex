% ----- 序 -----
\chapter*{前言}
\addcontentsline{toc}{chapter}{前言}
\markboth{前言}{} % 设置页眉

欢迎阅读本书的第2版。这本书致力于对TCP/P 协议族进行详细了解。不仅描述协议
如何操作,还使用各种分析工具显示协议如何运行。这可以帮助你更好地了解协议背后的设
计决策,以及它们如何相互影响。同时为你揭露协议的实现细节,而不需要你阅读实现的软
件源代码,或者设置一个实验性的实验室。当然,阅读源代码或设置一个实验室将不只是有
助于加深你的理解。

网络在过去30年中已经发生了巨大的变化。Internet 最初作为一个研究项目和令人好奇
的对象,现在已经成为一个全球性的通信设施,并被各国政府、企业和个人所依赖。TCP/
IP 协议族定义了 Internet 中每个设备交换信息的基本方法。经过十多年的发展,Internet 和
TCP/P 自身正在向兼容 IPv6的方向进化。在整本书中,我们将讨论 IPv6 和目前的IPv4,
着重关注它们之间的重要不同点。遗憾的是,它们不直接进行互操作,因此需要关心和注意
其演变的影响。

本书的读者对象是希望更好地了解当前的TCP/IP协议族以及它们如何运作的人员:网
络操作员和管理员、网络软件开发人员、学生,以及需要掌握TCP/IP 的用户。我们提供的
材料包括读者感兴趣的新材料和第1版已有的材料,希望读者能从中找到有用和有趣的新旧
材料。

\section*{第1版的评论}

距本书第1版出版已过去近20年。对于希望了解 TCP/IP 协议细节的学生和专业人士而
言,本书仍然是一个宝贵的资源,这些细节在许多其他同类教材中是难以获得的。目前,它
仍是有关TCP/IP 协议运行的详细信息的最好参考。但是,即使是信息和通信技术领域量好
的书籍,经过一段时间之后也会过时,当然本书也不例外。在这个版本中,我希望通过引入
新材料来彻底更新 Stevens 博士的前期工作,同时能够保持前作的极高水准和对其很多书籍
都包含的知识的详细介绍。

第1版涵盖了各种类型的协议和它们的操作,范围从链路层到应用和网络管理的所有
方面。目前,将如此广泛的材料综合在一卷中篇幅将会很长。因此,第2版特别关注核心协
议:那些级别相对较低的协议,常用于为 Internet 提供配置、命名、数据传输和安全等基础
性服务。关于应用、路由、Web 服务和其他重要主题被放到后续卷中。

从第1版出版以来,对TCP/IP 相应规范的实现在鲁棒性和规范性方面的改进已取得相
当大的进展。第1版中很多例子出现明显的实现错误或不符合要求的行为,这些问题在当前
可用的系统中已经得到解决,至少对于IPv4如此。考虑到在过去18年中 TCP/IP 协议的应
用日益广泛,这个事实并不令人吃惊。不符合要求的实现是比较罕见的,这证明了协议族整
体是比较成熟的。当前,在核心协议的运行中遇到的问题,通常涉及不常使用的协议功能。
在第1版中不太关注的安全问题,在第2版中花费了相当的笔墨来讨论。

\section*{21世纪的互联网环境}
Internet 使用模式和重要性自第1版出版以来已经发生了很大变化。最明显的具有分水
岭意义的事件是万维网在20世纪90年代初的建立和随后开始的激烈的商业化。这个事件大
大加快了大量有不同目的(有时冲突)的人对Internet 的使用。因此,这个最初实现在一个
小规模的学术合作环境中的协议和系统已受限于有限的可用地址,并且需要增加安全方面的
考虑。

为了应对安全威胁,网络和安全管理员纷纷为网络引入专门的控制单元。无论是大型企
业还是小型企业和家庭,现在常见的做法是将防火墙布置在 Internet 的连接点。随着过去十
年IP 地址和安全需求的增长,网络地址转换(NAT)现在几乎被所有路由器支持,并且得到
广泛的使用。它可以缓解地址短缺的压力,允许站点从服务提供商(对每个同时在线的用户)
获得一组相对较少的可路由的Internet 地址,无须进一步协调就可以为本地主机提供大量的
地址。部署 NAT 的结果是减缓了向IPv6(它提供了几乎不可思议的大量地址)的迁移,解决
了一些旧协议的互操作性问题。

随着PC用户在20世纪90年代中期开始要求连接 Internet,最大的PC软件供应商
(微软)放弃了其原来只提供专用Internet协议的策略,转而努力在自己的大部分产品中兼
容 TCPAIP。此后,运行 Windows 操作系统的PC变接人 Internet 的主体。随着时间的推
移,基于Linux 系统的主机数量显著上升,意味着这种系统现在有可能取代微软的领跑者地
位。其他操作系统,包括 Oracle 的 Solaris 和 Berkeley 的基于BSD 的系统,曾经代表了接人
Internet 的大多数系统,而现在只占一小部分。苹果的OS X操作系统(基于Mac)已成为一
个新的竞争者并日益普及,特别是在便携式计算机用户中。2003年,便携式计算机(笔记本
电脑)的销量超过了台式机,成为个人电脑销售的主力类型,它们的快速增长带来了对支持
高速上网的无线基础设施的需求。根据预测,2012年以后访问 Internet 的最常用方法是能
手机。平板电脑也是一个快速增长的重要竞争者的代表。

现在有大量场所提供了无线网络,例如餐厅、机场、咖啡馆,以及其他公共场所。
它们通常使用办公或家庭环境的局域网设备,提供短距离、免费或低费用、高速、无线
Internet 连接。一系列基于蜂窝移动电话标准(例如 LTE、UMTS、HSPA、EV-DO)的“无
线宽带”替代技术已广泛用于世界发达地区(一些发展中地区争相采用较新的无线技术),
为了提供更大范围的运营,通常需要在一定程度上减少带宽和降低基于流量的定价。两
种类型的基础设施满足了用户使用便携式计算机或更小的设备在移动过程中访问 Internet
的需要。在任何情况下,移动终端用户通过无线网络访问 Internet 都会带来两个对TCP/
IP 协议体系结构的技术挑战。首先,移动性影响了 Internet 的路由和寻址结构,打破了
主机基于附近的路由器分配地址的假设。其次,无线链路可能因更多原因而断开并导致数
据丢失,这些原因与典型的有线链路(通常不会丢失太多数据,除非网络中有太多流量)
不同。

最后,Internet 已经促进了由对等应用形成的“覆盖”网络的兴起。对等应用不依赖于
中心服务器完成一项任务,而是通过一组对等计算机之间的通信和交互完成一项任务。对等
计算机可以由其他终端用户来操作,并且可能快速进入或离开一个固定的服务器基础设施。
“覆盖”的概念刻画了如下事实:由这些交互的对等方形成一个网络,并且覆盖在传统的基
于TCP/IP 的网络上(在低层的物理链路之上实现覆盖)。对于那些对网络流量和电子商务有.
浓厚兴趣的研究者而言,对等应用的发展没有对卷1中所描述的核心协议产生深远的影响,
但是覆盖网络的概念在网络技术研究中晋遍受到重视。

\section*{第2版的内容变化}
第2版的最重要的变化是对第1版全部内容的整体重组和安全方面材料的显著增加。第
2版没有尝试覆盖 Internet的每个层次中使用的所有常用协议,而是关注正在广泛使用的非
安全的核心协议,或者预计在不久的将来广泛使用的协议:以太网(802.3)、Wi-Fi(802.11)、
PPP、ARP、IPv4、IPv6、TCP、UDP、DHCP 和DNS。系统管理员和用户可能都会用到这
些协议。

第2版通过两种方法来讨论安全性。首先,每章中都有专门的一节,用于介绍对本章所
描述协议的已知攻击和对策。这些描述没有介绍攻击的方法,而是提示了协议实现(或规范,
在某些情况下)不够健全时可能出现的问题。在当前的 Internet 中,对于不完整的规范或不
健全的实现,即使是相对简单的攻击,也可能导致关键的任务系统受到损坏。

第二个重要的安全性讨论出现在第18章,对安全和密码学中的一些细节进行研究,包
括协议,例如IPsec、TLS、DNSSEC 和DKIM。目前,这些协议对希望保持完整性或安全操
作的任何服务或应用的实现都是非常重要的。随着Internet 在商业上的重要性的增加,安全
需求(以及威胁的数量)已成比例增加。

虽然IPv6没有被包括在第1版中,但是未分配的IPv4地址块在2011年2月已耗尽,
现在有理由相信IPv6的使用可能会显著加快速度。IPv6 主要是为了解决IPv4地址耗尽问题,
随着越来越多的小型设备(例如移动电话、家用电器和环境传感器)接入 Internet,IPv6 正在
变得越来越重要。如世界 IPv6日(2011年6月8日)这种事件有助于表明Internet 可以继续
工作,即使是对底层协议进行重大修改和补充。

对第2版结构变化的第二个考虑是淡化那些不再常用的协议,以及更新那些自第1版出
版以来已大幅修订的内容。那些涉及 RARP、BOOTP、NFS、SMTP 和SNMP的章节已从书
中删除,SLIP 协议的讨论已被废弃,而DHCP 和 PPP(包括 PPPoE)的讨论篇幅被扩大。IP
转发(第1版中的第9章)功能已被集成在这个版本的第5章的IPv4 和 IPv6协议的整体描
述中。动态路由协议(RIP、OSPF、BGP)的讨论已被删除,因为后两个协议都应该单独通
过一本书来讨论。从ICMP开始到 IP、TCP 和 UDP,针对IPv4与IPv6 操作上差异明显的情
况,对每种操作的影响进行了讨论。这里没有专门针对IPv6的一章,而是在合适位置说明
它对每个现有的核心协议的影响。第1版中的第15 章和第25~30章,致力于介绍 Internet
应用和它们的支持协议,其中的大部分章节已删除,仅在必要时保留对底层的核心协议操作
的说明。

多个章节添加了新内容。第1章从网络问题和体系结构的常规介绍开始,接着是对
Internet 进行具体介绍。第2章涵盖 Internet 的寻址体系结构。第6章是新的一章,讨论主机
配置和在系统中如何“显示”网络。第7章介绍了防火墙和网络地址转换(NAT),包括 NAT
如何用于可路由和不可路由的地址空间。第1版所用的工具集已被扩大,现在包括 Wireshark
(一个免费的具有图形用户界面的网络流量监控应用程序)。

第2版的目标读者与第1版保持一致。阅读本书不需要具备网络概念的先期知识,但高
级读者可以从细节和参考文献中获得更大收获。每章包括一份丰富的参考文献集,供有兴趣
的读者查看。

\section*{第2版的编辑变化}
第2版中内容的整体组织流程仍然类似于第1版。在介绍性的内容(第1 章和第2章)
之后,采用自底向上方式介绍 Internet 体系结构中涉及的协议,以说明前面提到的网络通信
是如何实现的。与第1版一样,捕获的真实数据包用于在适当的位置说明协议的操作细节。
自第1版出版以来,免费的图形界面的数据包捕获和分析工具已经问世,它们扩展了第1
版中使用的tcpdump 程序的功能。在第2版中,如果基于文本的数据包捕获工具的输出很
容易解释,就使用 tcpdump。但是,在大多数情况下,使用 Wireshark 工具的屏幕截图。需
要注意的是,为了清楚地说明问题,有些输出列表(包括tcpdump 输出的快照)经过包装或
简化。

数据包跟踪内容说明了本书封二所描述的网络的一个或多个部分的行。它代表了一个
宽带连接的“家庭” 环境(通常用于客户端访问或对等网络)、一个“公共”环境(例如咖啡
厅)和一个企业环境。在例子中使用的操作系统包括 Linux、Windows、FreeBSD 和 Mac Os
X。目前,各种操作系统及不同版本被用于 Internet 中。

每章的结构相对第1版已稍作修改。每章开头是对该章主题的介绍,接着是历史记录
(在某些情况下),然后是本章详细资料、总结和一组参考文献。在大多数章中,章末都描述
了安全问题和攻击。每章的参考文献体现了第2版的变化。它们使得每章更具自包含性,读
者几乎不需要“长距离页面跳转”就能找到参考文献。有些参考文献通过增加网址来提供更
容易的在线访问。另外,论文和著作的参考文献格式已变为一种相对更紧凑的格式,包括每
个作者姓氏的首字母和一个两位数表示的年(例如,以前的[Cerf and Kahn 1974] 现在缩短
为[CK74])。对于使用的众多RFC参考文献,用RFC编号代替了作者姓名。这样做遵循了
典型的RFC规范,并将所有引用的RFC集中在参考文献列表中。

最后说明的是,继续保持本书的印刷惯例。但是,我们选择使用的编辑和排版格式,与
Stevens 博士和 Addison-Wesley Professional Computing Series 系列丛书的其他作者使用的
Troff系统不同。因此,最后的审稿任务利用了文字编辑 Barbara Wood 的专业知识。我们希
望大家很高兴看到这个结果。

\begin{flushright}
  \emph{Kevin R. Fall}
\end{flushright}
\begin{flushright}
  \emph{Berkeley,California}
\end{flushright}
\begin{flushright}
  \emph{2011年9月}
\end{flushright}
