% ----- 序 -----
\chapter*{序}
\addcontentsline{toc}{chapter}{序}
\markboth{序}{} % 设置页眉
% \lipsum[1-3]

读者很少能找到这样一本历史和技术全面且非常准确地讨论众所周知主题的书籍。我佩
服这项工作的原因之一是它给出的方案都让人信服。TCP/IP体系结构在构思时就是一个产
品。在适应多方面呈百万倍或以上不断增长的需求,更不用说大量的应用程序方面,它是非
凡的。理解体系结构的范围和局限性以及它的协议,可以为思考未来的演变甚至革命奠定良
好的基础。

在早期互联网体系结构的制定中,“企业”的概念并没有被真正认识。因此,大多数网
络都有自己的IP 地址空间,并在路由系统中直接“公布”自己的地址。在商业服务被引入
之后,Internet 服务提供商像中介那样“公布”自己客户的 Internet地址块。因此,大多数的
地址空间被分配为“提供商依赖”方式。“提供商独立”编址很少见。这种网络导致路由汇
聚和全球路由表大小的限制。虽然这种方式有好处,但它也带来了“多归属” 问题,这是由
于用户的提供商依赖地址在全球路由表中没有自己的条目。IP 地址“短缺”也导致了网络地
址转换,它也没有解决提供商依赖和多归属问题。

通过阅读这本书可以唤起读者对复杂性的好奇—一这种复杂性由工作在几种网络和应
用环境下的一组相对简单的概念发展而来。当各章展开时,读者可以看见复杂性程度随着日
益增长的需求而变化—这部分是由新的部署情况和挑战决定的,系统规模的激增就更不必
说了。

“企业”用户网络安全的问题迫使人们使用防火墙提供边界安全。这样做虽然有用,但
是很明显对本地Internet 基础设施的攻击可以通过内部(例如将一台受感染的计算机放入内
部网络,或用一个受感染的便携驱动器通过USB 端口感染一台内部计算机)进行。

很明显,除了需要通过引入 IPv6 协议(它有340x103个地址)扩大 Internet 地址空间
之外,还强烈需要引入各种安全增强机制,例如域名系统安全扩展(DNSSEC)等。

究竟是什么使这本书看起来很独特?据我估计是对细节的重视和对历史的关注。它提供
了解决已经演变的网络问题的背景和意义。它在确保精确和揭露剩余问题方面不懈努力。对
于一位希望完善和确保Internet 操作安全,或探索持续存在的问题的其他解决方案的工程师
来说,这本书所提供的见解将是非常宝贵的。作者对当前 Internet 技术的彻底分析是值得称
赞的。

\begin{flushright}
    \emph{Vint Cerf}
\end{flushright}

\begin{flushright}
    \emph{Woodhurst}
\end{flushright}

\begin{flushright}
    \emph{2011年6月}
\end{flushright}

